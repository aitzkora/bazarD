\documentclass[11pt,a4wide]{article}

\usepackage{amsmath,amssymb,amsfonts}
\usepackage{graphicx}
\newcommand{\tiret}{\rule[0.6ex]{1.3ex}{0.22ex}}

% !! pour le francais
\usepackage[utf8]{inputenc}
\usepackage[english]{babel}
\usepackage{listings}
\usepackage{relsize}
\usepackage{color}
\usepackage{fancyvrb}
\usepackage{punk}
\usepackage{emerald}

\newcommand{\fleche}{\alert{$\pmb{\longrightarrow}$}~~}
\newcommand{\calM}{{\mathcal{M}}}


%%%%% for smiley
\def\mysmile#1{{\ooalign{\hfil\lower.06ex % a smiley face
 \hbox{$\scriptscriptstyle#1$}\hfil\crcr
 \hfil\lower.7ex\hbox{\"{}}\hfil\crcr
 \mathhexbox20D}}}


\newcommand{\bad}{\bf \textcolor{red}{\mysmile \frown}}
\newcommand{\neutral}{\bf \textcolor{blue}{\mysmile \minus}}
\newcommand{\good}{\bf \textcolor{green}{\mysmile \smile}}

\newcommand{\calU}{{\mathcal{U}}}
\newcommand{\Cau}{\mathcal{C}_{n+1}}
\newcommand{\R}{\mathbb{R}}
\newcommand{\Sym}[1]{\mathcal{S}_{#1}(\R)}
\newcommand{\tran}{^{\top}}
\newcommand{\pssg}{\langle \langle}
\newcommand{\pssd}{\rangle \rangle}
\newcommand{\aronde}{\mathcal{A}}
\newcommand{\Tr}{\mathtt{Tr}}
\newcommand{\accol}[1]{{\left\{ \begin{array}{ll} #1 \end{array} \right.}}
\newcommand{\prods}[2]{\langle #1,#2 \rangle}
\newcommand{\Aa}{\mathcal{A}}
\newcommand{\norm}[1]{\|#1\|}
\newcommand{\matlab}{{\ECFTeenSpirit Malab}\ }
\newcommand{\python}{{\bf Python}\ }
\newcommand{\numpy}{{\bf Numpy}\ }
\newcommand{\scipy}{{\bf Scipy}\ }
\definecolor{dkgreen}{rgb}{0,0.6,0}
\definecolor{gray}{rgb}{0.5,0.5,0.5}
\definecolor{mauve}{rgb}{0.58,0,0.82}
\lstset{ %
  language=python,                % the language of the code
  framerule=0pt,
  basicstyle=\relsize{-2}\ttfamily,  % the size of the fonts that are used for the code
  %backgroundcolor=\color{black!10},  % choose the background color. You must add \usepackage{color}
  showspaces=false,               % show spaces adding particular underscores
  showstringspaces=false,         % underline spaces within strings
  showtabs=false,                 % show tabs within strings adding particular underscores
  %frame=single,                   % adds a frame around the code
  rulecolor=\color{black},        % if not set, the frame-color may be changed on line-breaks within not-black text (e.g. commens (green here))
  breakatwhitespace=false,        % sets if automatic breaks should only happen at whitespace
  keywordstyle=\color{blue},      % keyword style
  commentstyle=\color{dkgreen},   % comment style
  stringstyle=\color{mauve}  
}

\newcommand{\guill}[1]{``#1''} % attention deja dans mycv


\title{Comment utiliser \python comme alternative à \matlab}
\begin{document}
\maketitle
\section{Introduction}
  Pour ce projet, nous allons principalement utiliser 2 modules (ou bibliothèques) d'extension de \python qui
  permettent de faire du calcul numérique \numpy et \scipy.

  \subsection{Quelques Recommendations}
 
    \begin{itemize}

      \item Eviter le plus possible les boucles $\Rightarrow$ penser vectoriel
      \item Allouer les structures de données avant d'écrire dedans : \lstinline!a=np.zeros(1000);!
      \item eventuellement désallouer la mémoire qui n'est plus utilisé (utiliser {\tt del}) ; de toute façon \python 
      a un ramasse-miettes intégré.
      \item Faire des imports avec des noms : \lstinline! import numpy as np!
       
    \end{itemize}

  \subsection{Quelques differences avec \matlab}
    \begin{itemize}
      
      \item Attention \python utilise des références : 
      \begin{lstlisting} 
      a = np.zeros(4); b=a; b[1] = 10; a[1]
      \end{lstlisting}
      renvoie 10 et non pas zéro!
      \item Pour copier un objet, on utilise {\tt .copy()} : ainsi 
      \begin{lstlisting} 
      a = np.zeros(4); b=a.copy(); b[1] = 10; a[1]
      \end{lstlisting}
      renvoie 0.0 
    \end{itemize}

\end{document}
